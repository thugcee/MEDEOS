\documentclass[a4paper]{article}
\usepackage{polski}
\usepackage[utf8]{inputenc}
\usepackage{mdwlist}
\usepackage{paralist}
\usepackage{listings}
\usepackage[usenames,dvipsnames]{color}
\usepackage[bookmarks=true]{hyperref}
\hypersetup{
    unicode=false,          % non-Latin characters in Acrobat’s bookmarks
    pdftoolbar=true,        % show Acrobat’s toolbar?
    pdfmenubar=true,        % show Acrobat’s menu?
    pdffitwindow=false,     % window fit to page when opened
    pdfstartview={FitH},    % fits the width of the page to the window
    pdfkeywords={keywords}, % list of keywords
    pdfnewwindow=true,      % links in new window
    colorlinks=true,       % false: boxed links; true: colored links
    linkcolor=black,          % color of internal links
    citecolor=black,        % color of links to bibliography
    filecolor=black,      % color of file links
    urlcolor=black           % color of external links
}
\usepackage{tikz}
\usetikzlibrary{shapes,arrows,shadows,trees} % for pgf-umlsd
\usepackage[underline=true,rounded corners=false]{pgf-umlsd}

\lstset{ %
basicstyle=\footnotesize,       % the size of the fonts that are used for the code
numbers=left,                   % where to put the line-numbers
numberstyle=\footnotesize,      % the size of the fonts that are used for the line-numbers
stepnumber=5,                   % the step between two line-numbers. 
numbersep=5pt,                  % how far the line-numbers are from the code
backgroundcolor=\color{white},  % choose the background color. You must add \usepackage{color}
showspaces=false,               % show spaces adding particular underscores
showstringspaces=false,         % underline spaces within strings
showtabs=false,                 % show tabs within strings adding particular underscores
%frame=single,	                % adds a frame around the code
tabsize=2,	                % sets default tabsize to 2 spaces
captionpos=b,                   % sets the caption-position to bottom
breaklines=true,                % sets automatic line breaking
breakatwhitespace=false,        % sets if automatic breaks should only happen at whitespace
title=\lstname,                 % show the filename of files included with \lstinputlisting; also try caption instead of title
%escapeinside={\%*}{*)},          % if you want to add a comment within your code
inputencoding=utf8,
extendedchars=true
}

%highlight
\usepackage{color}
\usepackage{alltt}
\usepackage{ucs}
\input {highlight.sty}

\begin{document}
\title{Medical Expertise Ordering System - User Interface}
\author{Seweryn Niemiec, Krzysztof Bogusławski,\\ 
Łukasz Martyniak, Bartosz Uchacz, Przemysław Wyrobek \\ 
\textbf{Academic Center of Computer Science } of \\ 
Westpomeranian University of Technology in Szczecin,\\
Fundacja IT}
\date{\today}
\maketitle
\tableofcontents

\section{Introduction}

Document which describes Medical Expertise Ordering System (MEDEOS, called later the
System) does not contain specification for user interface. User interface is not integral
part of the System and its functionality, design, method of presentation and level of
integration with other IT systems can vary greatly by implementation and depends on
individual users needs. To get the big picture of how the System works and provide some
idea of user interface construction, a simple case studies has been described here.

This document is a supplement to document titled ''Medical Expertise Ordering System''
and contains references to keywords defined there.

\section{An example customer -- consultant session}

This description is based on concrete case with many System's variables preselected. For
example, images created during examination are saved in PACS, but it doesn't mean that
PACS is required by the System. To get information about what is optional/variable and
what is required check the main document.

\subsection{Customer side}

\subsubsection{Sending new order}

\begin{enumerate}
  \item Radiology technician carries out an examination and DICOM image files are created
  as a result of it. Image files containing metadata are saved in PACS. Those
  metadata will be later used by the \emph{customer interface} to select images for a
  given person and examination. 
 \item The Customer (probably a physician) starts the \emph{customer interface} and from
 main menu selects \emph{create new order} option.
 \item Using the new order form, the customer selects images from PACS. Metadata from
 first selected image are used to fill some form fields (like person name), so the
 customer do not have to enter them again. Customer fills other fields in
 order to complete the order. One of the most important fields is the one which allows
 selection of a consultant. Customer selects a consultant from a predefined list.
 \item When the form is complete and all attachments selected, the customer selects 
 \emph{send order} option.
 \item If encryption or digital signing is used and private keys are password protected 
then input dialog should pop-up and allow customer to enter them.
 \item Order sending should be performed by the client in the background, not
 blocking the user interface.
\end{enumerate}

\subsubsection{Checking order state}

\begin{enumerate}
  \item From the main menu the customer selects \emph{awaiting orders} option.
  \item The list of sent orders shows up. It is divided/sorted by 3 groups: 
	\begin{inparaenum}[\itshape 1\upshape)]
		\item orders sent, uncompleted, waiting for customer action (consultant submitted
		comments to the order),
		\item orders sent, uncompleted, waiting for consultant action
		\item orders which sending is in progress.
	\end{inparaenum}.
  \item Customer selects interesting orders from the list to check its details (like
  transfer progress, comments from consultant, files list on the server, etc.)
\end{enumerate}

\subsubsection{Modifying an order equipped with a consultant comments}

\begin{enumerate}
  \item From the main menu the customer selects \emph{awaiting orders} option.
  \item The list of sent orders shows up. There are orders equipped with a consultant
  comments among other uncompleted orders.
  \item By selecting an order form the list the customer displays order's
  details and can read consultant comments.
  \item The customer makes changes in an order suitable to consultant's comments. He can
  provide additional text in order body or add new image files. All changes are 
  remembered by the customer interface.
  \item The customer selects \emph{send changes} option, to send all changes to the
  server. 
  \item The order is moved to the ''sending in progress group''.
\end{enumerate}

\subsection{Consultant side}

\subsubsection{Making an interpretation}

Two example solutions for software at consultant side are shown below. 

\begin{enumerate}
  \item When complete order is stored on the server, its image files are automatically
  sent directly to preconfigured diagnostic station. A consultant (a radiologist)
  using diagnostic station makes an examination interpretation, which is then entered to
  the System via consultant interface (potentially run on another computer).
  \item Consultant or technician using consultant interface browses orders and
  selects some of them for interpretation. Radiographic images contained in those orders
  are send to manually selected diagnostic station. Further actions same as in point 1.
\end{enumerate}

Consultant interface can have direct access to files (or database) on a server or connect
to a server via HTTP[S] and work using GET, PUT, DELETE. The second solution requires
implementation of distinct access profiles on a server for consultant and customer
clients.

\section{Methods of integration with existing IT systems}

The System itself does not (or at least does not have to) interact with existing IT
systems, its customer and consultant interfaces' job. Here is few possible scenarios:
\begin{description}
  \item[no integration]\hfill\\ interface do not talk with PACS, HIS, RIS; files
  which will be attached to an order have to be transported to a computer running
  interface by other means,
  \item[integration via DICOM]\hfill\\ interface has access to radiography files
  via DICOM communication protocol,
  \item[integration via DICOM and with HIS and RIS]\hfill\\ interface has access to
  radiography files via DICOM communication protocol and is exchanging metadata with
  HIS/RIS via its own, suitable for a given system mechanism; orders are
  synchronised between the System and HIS/RIS,
  \item[integration via DICOM and built-in into HIS or RIS]\hfill\\
  interface is integral part of HIS/RIS and has access to radiography files via DICOM 
  \item[built-in into PACS]\hfill\\ interface is a integral part of PACS.
\end{description}

\end{document}
