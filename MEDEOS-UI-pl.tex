\documentclass[a4paper]{article}
\usepackage{polski}
\usepackage[utf8]{inputenc}
\usepackage{mdwlist}
\usepackage{paralist}
\usepackage{listings}
\usepackage[usenames,dvipsnames]{color}
\usepackage[bookmarks=true]{hyperref}
\hypersetup{
    unicode=false,          % non-Latin characters in Acrobat’s bookmarks
    pdftoolbar=true,        % show Acrobat’s toolbar?
    pdfmenubar=true,        % show Acrobat’s menu?
    pdffitwindow=false,     % window fit to page when opened
    pdfstartview={FitH},    % fits the width of the page to the window
    pdfkeywords={keywords}, % list of keywords
    pdfnewwindow=true,      % links in new window
    colorlinks=true,       % false: boxed links; true: colored links
    linkcolor=black,          % color of internal links
    citecolor=black,        % color of links to bibliography
    filecolor=black,      % color of file links
    urlcolor=black           % color of external links
}
\usepackage{tikz}
\usetikzlibrary{shapes,arrows,shadows,trees} % for pgf-umlsd
\usepackage[underline=true,rounded corners=false]{pgf-umlsd}

\lstset{ %
basicstyle=\footnotesize,       % the size of the fonts that are used for the code
numbers=left,                   % where to put the line-numbers
numberstyle=\footnotesize,      % the size of the fonts that are used for the line-numbers
stepnumber=5,                   % the step between two line-numbers. 
numbersep=5pt,                  % how far the line-numbers are from the code
backgroundcolor=\color{white},  % choose the background color. You must add \usepackage{color}
showspaces=false,               % show spaces adding particular underscores
showstringspaces=false,         % underline spaces within strings
showtabs=false,                 % show tabs within strings adding particular underscores
%frame=single,	                % adds a frame around the code
tabsize=2,	                % sets default tabsize to 2 spaces
captionpos=b,                   % sets the caption-position to bottom
breaklines=true,                % sets automatic line breaking
breakatwhitespace=false,        % sets if automatic breaks should only happen at whitespace
title=\lstname,                 % show the filename of files included with \lstinputlisting; also try caption instead of title
%escapeinside={\%*}{*)},          % if you want to add a comment within your code
inputencoding=utf8,
extendedchars=true
}

%highlight
\usepackage{color}
\usepackage{alltt}
\usepackage{ucs}
\input {highlight.sty}

\begin{document}
\title{Medical Expertise Ordering System \\Interfejs użytkownika}
\author{Krzysztof Bogusławski, Seweryn Niemiec\\ 
\textbf{Akademickie Centrum Informatyki} \\ 
Zachodniopomorski Uniwersytet Technologiczny w Szczecinie\\
oraz Fundacja IT}
\date{\today}
\maketitle
\tableofcontents

\section{Wprowadzenie}

Dokument opisujący Medical Expertise Ordering System (w skrócie MEDEOS) nie zawiera
informacji na temat interfejsu użytkownika. Interfejs użytkownika nie stanowi integralnej części
Systemu i jego funkcjonalność, wygląd oraz sposób realizacji mogą być bardzo odmienne w
różnych wdrożeniach. Jednak dla lepszego zrozumienia działania całości stworzony został
opis przykładowego interfejsu użytkownika.

Dokument ten jest dodatkiem do dokumentu pt. ,,Medical Expertise Ordering System'' i odnosi się
do zdefiniowanych tam haseł. 

Dla lepszego zobrazowania całości (System + interfejsy użytkowników) opis
funkcjonalności interfejsu użytkownika został przedstawiony za pomocą analizy
wyimaginowanych przypadków.

\section{Przykładowe sesje pracy}

\subsection{Strona zleceniodawcy}

\subsubsection{Wysyłanie zlecenia}

\begin{enumerate}
  \item Technik radiologii wykonuje badanie, którego wynikiem jest seria obrazów. Obrazy
  wraz z metadanymi zostają zapisane w systemie PACS. Metadane zawierają informacje
  pozwalające powiązać obrazy z pacjentem i z tym konkretnym badaniem (mogą być też użyte
  dowolne znaczniki używane wewnętrznie przez system szpitalny, za pomocą których, te
  zdjęcia będą mogły być odnalezione poprzez DICOM).
 \item Zleceniodawca uruchamia interfejs konsultanta i z menu głównego wybiera opcję
 \emph{nowe zlecenie}.
 \item Formularz nowego zlecenia pozwala mu wybrać zdjęcia z systemu PACS. Dzięki
 metadanym pobranym z pierwszego z wybranych zdjęć, wstępnie wypełniane są pola
 formularza, tak aby technik nie musiał ponownie wszystkiego wprowadzać. Zleceniodawca ma
 możliwość wprowadzenia poprawek do automatycznie wypełnionych pół oraz wypełnienia
 pustych pól. Podczas edycji formularza ma możliwość edycji listy załączonych
 obrazów (lub innych dokumentów).
 \item Kiedy formularz jest kompletny i wszystkie załączniki wybrane, zleceniodawca
 wybiera opcję \emph{wyślij zlecenie}.
 \item Jeśli do uwierzytelniania klienta po HTTPS używany jest klucz prywatny
 zabezpieczony hasłem, to pojawia się okno dialogowe z pytaniem o hasło. 
 \item Procedura wysyłania zlecenia odbywa się w tle, nie blokując interfejsu
 użytkownika. Wysyłaniem zlecenia zajmuje się program klienta Systemu.  
\end{enumerate}

\subsubsection{Sprawdzanie stanu zlecenia}

\begin{enumerate}
  \item Zleceniodawca w menu głównym wybiera opcję \emph{zlecenia oczekujące}.
  \item Pojawia się lista wysłanych zleceń oczekujących na opis i zleceń w trakcie
  wysyłania. Lista jest podzielona na 3 grupy: 
	\begin{inparaenum}[\itshape 1\upshape)]
		\item zlecenia wysłane, niezrealizowane, oczekujące na akcję ze strony wysyłającego
		(konsultant zgłosił komentarz do zlecenia),
		\item zlecenia wysłane, niezrealizowane, oczekujące na akcję ze strony konsultanta oraz
		\item zlecenia w trakcie wysyłania.
	\end{inparaenum} Każda grupa jest wyraźnie oznaczona kolorami.
  \item Wybierając zlecenie z listy, konsultant ma możliwość sprawdzenia szczegółów
  zlecenia (np. przeczytać komentarz konsultanta lub sprawdzić stan postępu wysyłania zlecenia)
\end{enumerate}

\subsubsection{Modyfikowanie zlecenia wysłanego, opatrzonego komentarzem}

\begin{enumerate}
  \item Zleceniodawca w menu głównym wybiera opcję \emph{zlecenia oczekujące}.
  \item Pojawia się lista wysłanych zleceń oczekujących, a wśród nich są zlecenia opatrzone
  komentarzem konsultanta.
  \item Wybierając zlecenie z listy, zleceniodawca sprawdzenia szczegóły zlecenia, między
  innymi treść komentarza.
  \item W odpowiedzi na komentarz zleceniodawca może zmienić listę załączonych plików.
  Interfejs zapamiętuje zmiany wprowadzane przez technika. Kiedy technik wybierze opcję
  \emph{prześlij zmiany}, interfejs zleceniodawcy przekazuje listę modyfikacji do
  klienta, a ten w tle przesyła zmiany na serwer. Zlecenie przechodzi do grupy zleceń w
  trakcie wysyłania.
\end{enumerate}

\subsection{Strona konsultanta}

\subsubsection{Opisywanie zlecenia}

Pośród wielu możliwych sposobów realizacji tego procesu na bazie Systemu, przedstawione
zostały dwa przykładowe. 

\begin{enumerate}
  \item Dane obrazowe zlecenia otrzymanego przez serwer są automatycznie przesyłane
  bezpośrednio na odpowiednią stację diagnostyczną. Konsultant (lekarz radiolog) wykonuje
  opis badania, który jest następnie wprowadzany poprzez interfejs konsultanta do Systemu.
  \item Konsultant lub osoba techniczna, korzystając z interfejsu konsultanta, przegląda
  odebrane zlecenia i decyduje o ich wysłaniu na odpowiednią stację diagnostyczną. Dalsza
  procedura jak w punkcie 1.
\end{enumerate}

Interfejs konsultanta może mieć bezpośredni dostęp do danych serwera (do katalogu
zawierającego pliki zleceń lub bazy danych) lub łączyć się z serwerem poprzez HTTP[S] i
działać na podobnej zasadzie jak klient zleceniodawcy. W tym drugim przypadku konieczne
jest zastosowanie odmiennej polityki dostępu dla dwóch rodzajów klientów: zleceniodawcy i
konsultanta. 

\section{Możliwe sposoby integracji z istniejącymi systemami szpitalnymi}

Sam System nie styka się z istniejącymi systemami szpitalnymi. Integracja dotyczy jedynie
interfejsów zleceniodawcy i konsultanta. Możliwe są następujące scenariusze:
\begin{description}
  \item[bez integracji]\hfill\\ interfejs użytkownika nie komunikuje się z systemami
  PACS, HIS, RIS; pliki które mają być dołączone do zlecenia muszą być ręcznie udostępnione dla
  programu realizującego interfejs,
  \item[z integracja poprzez DICOM]\hfill\\ interfejs użytkownika ma dostęp do danych
  obrazowych poprzez protokół DICOM,
  \item[z integracja poprzez DICOM i z HIS lub RIS]\hfill\\ interfejs użytkownika ma
  dostęp do danych obrazowych poprzez protokół DICOM oraz do HIS/RIS poprzez własny, odpowiedni dla danego
  systemu mechanizm; istnieje synchronizacja danych między interfejsem użytkownika Systemu,
  a HIS/RIS,
  \item[z integracja poprzez DICOM i wbudowanie w HIS lub RIS]\hfill\\ interfejs
  użytkownika jest integralną częścią HIS/RIS i ma dostęp do danych obrazowych poprzez protokół DICOM 
  \item[z integracja z PACS]\hfill\\ interfejs użytkownika jest integralną częścią PACS. 
\end{description}

\end{document}
